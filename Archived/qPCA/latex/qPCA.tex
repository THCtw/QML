\documentclass[a4paper]{article}

\usepackage{amsmath}
\usepackage{physics}
\usepackage{hyperref}
\usepackage[a4paper,left=3cm,right=3cm,top=2.5cm,bottom=2.5cm]{geometry}

\title{Quantum Principal Component Analysis}
\author{Tzu Hsuan Chang}

\begin{document}
\maketitle

\section{Introduction}
\label{sec:Intro}


\subsection*{Density Matrix Exponentiation}
\label{subsec:dme}
    \begin{equation} \label{eq1}
    \setlength{\jot}{10pt}
    \begin{aligned}
        Tr_P ( e^{-i \hat{S} \Delta t} \rho  \otimes \sigma e^{i \hat{S} \Delta t} )
            &= Tr_P [ \, ( \cos \Delta t + i \hat{S} \sin \Delta t ) \ \rho \otimes \sigma \ ( \cos \Delta t - i \hat{S} \sin \Delta t ) ] \, \\
            &\approx Tr_P [ \, \rho \otimes \sigma - i \Delta t [\rho, \sigma] + O(\Delta t^2) ] \, \\
            &= \sigma - i \Delta t [\rho, \sigma] + O (\Delta t^2) \\
            &\approx e^{-i \rho \Delta t} \ \sigma \ e^{i \rho \Delta t}
    \end{aligned}
    \end{equation}


Where $Tr_P$ is the partial trace $\rho$ over the first variable and $\hat{S}$ is the swap operator. Note that $\rho \otimes \sigma \hat{S} = \rho \otimes \sigma$. By repeating \eqref{eq1} n times,
    \begin{equation} \label{eq2}
    \setlength{\jot}{10pt}
    \begin{aligned}
        &Tr_{P_1} Tr_{P_2} ... Tr_{P_n} [ \, e^{-i \hat{S}_n \Delta t} e^{-i \hat{S}_{n-1} \Delta t} ... e^{-i \hat{S}_1 \Delta t} ( \rho^{\otimes n}  \otimes \sigma ) e^{i \hat{S}_1 \Delta t} ... e^{i \hat{S}_{n-1} \Delta t} e^{i \hat{S}_n \Delta t} ] \, \\
        = \ &Tr_{P_n} [ \, \ 
            e^{-i \hat{S}_n \Delta t} \ 
            Tr_{P_{n-1}} [\, 
            e^{-i \hat{S}_{n-1}}
                (... Tr_{P_1} [ \, e^{-i \hat{S}_1 \Delta t} ( \rho^{\otimes n}  \otimes \sigma ) e^{i \hat{S}_1 \Delta t} ] \,)  
                    e^{i \hat{S}_{n-1}}
                ] \, \
                e^{i \hat{S}_n \Delta t} \ 
        ] \, \\
        \approx \ &e^{-i \rho n \Delta t} \ \sigma \ e^{i \rho n \Delta t}
    \end{aligned}
    \end{equation}


\section{Two qubits case}
\label{sec:2qcase}
Here, we use the quantum algorithm in Ref.\cite{coles2018quantum} to perform PCA on a two features case, where the covariance matrix can be  represented by a single qubit. The covariance matrix $\Sigma$ and eigenvalues, $e_1$ and $e_2$, of the raw data provided in Ref.\cite{coles2018quantum} are,
\begin{equation} \label{eq:eq3}
	\Sigma = 
	\begin {pmatrix}
		0.380952   &   0.573476 \\
		0.573476   &   1.29693
	\end {pmatrix},
	e_1   =   1.57286,
	e_2   =   0.105029.
\end{equation}
Now we are going to calculate $e_1$ and $e_2$ using quantum algorithm. 

The first step is to calculate the $\hat{U}_{prep}$ classically which can prepare the two qubits pure state $\ket{\psi}$ from $\ket{0}$, where $\ket{\psi}$ is the purified state of $\Sigma$.

The second step is to prepare two copies of $\ket{\psi}$ on quantum computer by $\hat{U}_{prep}$. Note that $\hat{U}_{prep}$ can be decomposed as:
\begin{equation} \label{eq:eq4}
	\hat{U}_{prep}   =   (\hat{U}_A \otimes \hat{U}_B)   CNOT_{AB}   (\hat{U}_A^{\prime} \otimes \hat{I}_B).
\end{equation}
Where $\hat{U}_i$ can be represented as,
\begin{equation} \label{eq:eq5}
	\begin {pmatrix}
		\cos(\theta/2)   &   -e^{i \lambda} \sin{\theta/2} \\
		e^{i \phi} \sin{\theta/2}   &   e^{i \lambda + \phi} \cos{\theta/2}
	\end {pmatrix}.
\end{equation}

The third step is to calculate the purity.


\bibliographystyle{abbrv}
\bibliography{references}


\end{document}